\documentclass[11pt,letterpaper]{article}
\usepackage[top=3cm, bottom=2cm, left=2cm, right=2cm, columnsep=20pt]{geometry}
\usepackage{pdfpages}
\usepackage{graphicx}
\usepackage{etoolbox}
\apptocmd{\sloppy}{\hbadness 10000\relax}{}{}
% \usepackage[numbers]{natbib}
\usepackage[T1]{fontenc}
\usepackage{ragged2e}
\usepackage[french]{babel}
\usepackage{listings}
\usepackage{color}
\usepackage{soul}
\usepackage[utf8]{inputenc}
\usepackage[export]{adjustbox}
\usepackage{caption}
\usepackage{amsmath}
\usepackage{amssymb}
\usepackage{float}
\usepackage{csquotes}
\usepackage{fancyhdr}
\usepackage{wallpaper}
\usepackage{siunitx}
\usepackage[indent]{parskip}
\usepackage{textcomp}
\usepackage{gensymb}
\usepackage{multirow}
\usepackage[hidelinks]{hyperref}
\usepackage{abstract}
\usepackage{svg}
\renewcommand{\abstractnamefont}{\normalfont\bfseries}
\renewcommand{\abstracttextfont}{\normalfont\itshape}
\usepackage{titlesec}
\titleformat{\section}{\large\bfseries}{\thesection}{1em}{}
\titleformat{\subsection}{\normalsize\bfseries}{\thesubsection}{1em}{}
\titleformat{\subsubsection}{\normalsize\bfseries}{\thesubsubsection}{1em}{}

\usepackage{xcolor}
\definecolor{codegreen}{rgb}{0,0.6,0}
\definecolor{codegray}{rgb}{0.5,0.5,0.5}
\definecolor{codepurple}{rgb}{0.58,0,0.82}
\definecolor{backcolour}{rgb}{0.95,0.95,0.92}
\lstdefinestyle{mystyle}{
    backgroundcolor=\color{backcolour},   
    commentstyle=\color{codegreen},
    keywordstyle=\color{magenta},
    numberstyle=\tiny\color{codegray},
    stringstyle=\color{codepurple},
    basicstyle=\ttfamily\footnotesize,
    breakatwhitespace=false,         
    breaklines=true,                 
    captionpos=b,                    
    keepspaces=true,                 
    numbers=left,                    
    numbersep=5pt,                  
    showspaces=false,                
    showstringspaces=false,
    showtabs=false,                  
    tabsize=2
}
\lstset{style=mystyle}

\usepackage[most]{tcolorbox}
\newtcolorbox{note}[1][]{
  enhanced jigsaw,
  borderline west={2pt}{0pt}{black},
  sharp corners,
  boxrule=0pt, 
  fonttitle={\large\bfseries},
  coltitle={black},
  title={Note:\ },
  attach title to upper,
  #1
}

%----------------------------------------------------

\setlength{\parindent}{0pt}
\DeclareCaptionLabelFormat{mycaptionlabel}{#1 #2}
\captionsetup[figure]{labelsep=colon}
\captionsetup{labelformat=mycaptionlabel}
\captionsetup[figure]{name={Figure }}
\newcommand{\inlinecode}{\normalfont\texttt}
\usepackage{enumitem}
\setlist[itemize]{label=\textbullet}

\begin{document}
\begin{titlepage}
\center

\begin{figure}
    \ThisULCornerWallPaper{.4}{Polytechnique_signature-RGB-gauche_FR.png}
\end{figure}
\vspace*{2 cm}

\textsc{\Large \textbf{PHS3910 --} Techniques expérimentales et instrumentation}\\[0.5cm]
\large{\textbf{Équipe : Lundi 03}}\\[1.5cm]

\rule{\linewidth}{0.5mm} \\[0.5cm]
\Large{\textbf{Écran tactile acoustique}} \\[0.2cm]
\text{Fiche technique du prototype}\\
\rule{\linewidth}{0.2mm} \\[2.3cm]

\large{\textbf{Présenté à}\\
  Jean Provost\\
  Lucien Weiss\\[2.5cm]
  \textbf{Par :}\\
  Émile \textbf{Guertin-Picard} (2208363)\\
  Philippine \textbf{Beaubois} (2211153)\\
  Marie-Lou \textbf{Dessureault} (2211129)\\
  Maxime \textbf{Rouillon} (2213291)\\[3cm]}

\large{\today\\
Département de Génie Physique\\
Polytechnique Montréal\\}

\end{titlepage}

%----------------------------------------------------

\tableofcontents
\pagenumbering{roman}
\newpage

\pagestyle{fancy}
\setlength{\headheight}{14pt}
\renewcommand{\headrulewidth}{0pt}
\fancyfoot[R]{\thepage}

\pagestyle{fancy}
\fancyhf{}
\renewcommand{\headrulewidth}{1pt}
\fancyhead[L]{\textbf{PHS3910}}
\fancyhead[C]{Fiche technique de l'écran tactile acoustique}
\fancyhead[R]{\today}
\fancyfoot[R]{\thepage}

\pagenumbering{arabic}
\setcounter{page}{1}

%----------------------------------------------------

\section{Description générale}

%%%%%%%%%%%%%%%%%%%%%%%%%%%%%%%%%%%%%%%%%%%%%%%%%%%%
% Idées :

% Nombre de notes, un seul octave, une note à la fois.

%%%%%%%%%%%%%%%%%%%%%%%%%%%%%%%%%%%%%%%%%%%%%%%%%%%%

Cette fiche technique présente les caractéristiques d'un piano construit avec un
écran tactile acoustique. Un capteur piezoélectrique, sur une plaque de plexiglas 
de 5 mm d'épaisseur, localise un impact par son onde sonore, pour permettre 
de jouer la note appropriée en temps réel. Cette plaque 
et ses dimensions sont présentées à la figure \ref{piano_fig}. Le piano peut jouer\
une seule gamme (12 notes), et est limité à ne pouvoir jouer qu'une seule note à 
la fois. L'acquisition de signal sonore se fait à une fréquence de 44100 Hz par un
ADC, donnant des échantillons de 32 bits. Le contenu fréquentiel n'est pas filtré,
donc des fréquences sonores de 0 Hz à 22000 Hz sont présentes. Le délai entre la frappe
et le son de la note est d'environ 200 ms,
dépendant de la puissance de l'ordinateur qui lit les données du capteur. La résolution,
qui informe de la grandeur qu'une note doit prende pour être distincte, ainsi que le
contraste, qui quantifie la différence entre les signaux, sont données dans la table de
spécifications.

\begin{figure}[H]
  \centering
  \includesvg[scale=1]{schema.svg}
  \caption{Schéma avec dimensions du prototype de piano tactile. Le capteur
  piezoélectrique a son centre de positionné à 176 mm du côté gauche et à 207 mm
  du bas de la plaque approximativement.}
  \label{piano_fig}
\end{figure}



\section{Spécifications}

dire pourquoi on a testé res et contraste pour d'autres facteurs que ceux du prototype réel (aka pour l'analyse de réduction de couts)
%%%%%%%%%%%%%%%%%%%%%%%%%%%%%%%%%%%%%%%%%%%%%%%%%%%%
% Temps de réponse moyen serait nice

% Res et contraste :
% nb de bits des échantillons du signal (int16, float32 hors de la carte)
% contenu fréquentiel
% fréquence d'échantillonage

% Forme de tableau
%%%%%%%%%%%%%%%%%%%%%%%%%%%%%%%%%%%%%%%%%%%%%%%%%%%%


\section{Rapports de tests}

%%%%%%%%%%%%%%%%%%%%%%%%%%%%%%%%%%%%%%%%%%%%%%%%%%%%
% description brève des tests et présentations de leurs résultats.

% si graphiques : plusieurs courbes et infos sur le même

% Tests :

% acquisition des résolutions et contrastes pour les 3 facteurs

% *** écrire en qqpart l'indépendance du param de nb de bits vs les autres ***
%%%%%%%%%%%%%%%%%%%%%%%%%%%%%%%%%%%%%%%%%%%%%%%%%%%%

Pour pouvoir réduire les coûts blabla... \textcolor{red}{TODO}
\subsection{Nombre de bits des échantillons du signal}
our modifier le nombre de bits artificiellement de chaque signal, le processus est assez simple.
Le nombre de bits disponibles doit être déterminé au début pour la précision. En effet, lorsque le nombre de bits est réduit,
il n'est pas possible de tous les utiliser pour la précision. Dans les faits, un bit est toujours réservé pour la description du signe (positif ou négatif).
Puisque la base 2 est utilisée dans ce prototype, le nombre de niveaux disponibles pour approximer
les données est le suivant :
$ 2^{nbr\_bit -1}$.
Ainsi, pour diminuer la précision des valeurs du signal au nombre de bits souhaité,
le signal sera multiplié par $\frac{ 2^{nbr_bit -1}}{2}$ pour normaliser le signal, ensuite chaque
valeur sera arrondie à son niveau le plus proche. Le signal sera alors divisé par $\frac{ 2^{nbr_bit -1}}{2}$ pour
revenir à son échelle d'origine. Avec cette méthode, les résultats suivants ont été obtenus et sont présentés dans le tableau qui suit.
\begin{table}[ht]
  \centering
  \begin{tabular}{l c c c c}
  \hline
  Dictionnaire & Résolution & Erreur Résolution & Contraste & Erreur Contraste \\ \hline
  1bit & 0.070452 & 8.284432 & -0.057599 & 113.5591 \\
  2bit & 0.036577 & 0.227103 & 0.120715 & 4.715503 \\
  3bit & 0.041791 & 0.059722 & 0.495011 & 2.076969 \\
  4bit & 0.040465 & 0.024991 & 0.563969 & 1.053042 \\
  6bit & 0.042056 & 0.006118 & 1.107812 & 0.269374 \\
  8bit & 0.041545 & 0.002079 & 1.154641 & 0.09054 \\
  16bit & 0.049873 & 0.001649 & 1.392218 & 0.08161 \\
  Original & 0.063902 & 0.000766 & 1.248192 & 0.01538 \\
  \hline
  \end{tabular}
  \caption{Tableau des résultats}
\end{table}
  
En analysant les données présentes dans ce tableau, on constate qu'en dessous de 6 bits,  
les valeurs de résolution et de contraste commencent à être nettement moins correctes.  
Cette affirmation peut être démontrée par la nette augmentation des incertitudes, ce qui  
rend les valeurs plus incertaines. Il est donc juste de déterminer qu'un signal traité  
avec moins de ressources que celles requises pour 6 bits aurait un impact néfaste sur  
l'efficacité du fonctionnement du piano. Cependant, cela nous permet également de remarquer  
qu'un prototype pourrait fonctionner avec moins de bits que le prototype actuel.  
Cela pourrait être bénéfique car il permettrait de faire fonctionner correctement un prototype  
utilisant moins de mémoire et donc moins de ressources. Cela pourrait permettre de réaliser un projet 
similaire mais a moindre couts. 

\subsection{Incertitudes}
L'incertitude associée à l'amplitude du signal, ou l'axe y, est celle qui provient de la résolution 
de l'ADC. La résolution de l'ADC, $\delta$, dépend du nombre de bits des échantillons, et peut être décrite par:
\[\delta=\frac{\Delta_{max}}{2^{n}},\]
où $\Delta_{max}$ est l'éventail possible des mesures, et $n$ est le nombre de bits. L'incertitude correspond
à la moitié de cette valeur, soit $\sigma_{res}=\delta/2$. L'incertitude sur la position de la source du signal
est approximée par la moitié de la largeur d'un doigt, $\sigma_{pos}\approx4\ mm$. La corrélation des signaux a été déterminée
en calculant le produit scalaire. On sait que pour une multiplication, la propagation de l'incertitude se calcule de la
manière suivante:
\begin{align*}
  \sigma_{z}=z \sqrt{\frac{\sigma_x}{x}^2+\frac{\sigma_y}{y}^2}.
\end{align*}
L'incertitude totale sur le produit scalaire est donc:
\begin{align*}
  \sigma_{tot}=\sqrt{\sum_i^{N} \sigma_{z_i}^2},
\end{align*}
où $N$ est le nombre de points évalués pour les échantillons. Pour déterminer l'incertitude sur les
paramètres du fit gaussien, tout en considérant l'impact des incertitudes en $x$ et $y$, la fonction \textit{scipy.ODR}
a été utilisée. Finalement, l'incertitude sur le contraste correspond directement à l'incertitude sur le paramètre de
l'amplitude ($\sigma_{A}$), et l'incertitude sur la résolution (FWHM) correspond à:
\begin{align*}
  \sigma_{FWHM}=\sqrt{2ln2}\ \sigma_{\sigma},
\end{align*}
où $\sigma_{\sigma}$ est l'incertitude sur l'écart-type, $\sigma$.



\section{Codes}

Les codes présentés ci-dessous ont été utilisés pour faire fonctionner le piano et pour faire les
multiples tests. Ces derniers peuvent nécessiter d'être adaptés pour pouvoir faire des tests
spécifiques, ou encore pour décider quelles notes peuvent être jouées par le piano.

\subsection{Programme pour modifier le contenu fréquentiel et la fréquence d'échantillonage}

\begin{lstlisting}[language=python]
import numpy as np
import matplotlib.pyplot as plt
import json
import os
from scipy.optimize import curve_fit

# Charger le fichier JSON (dictionnaire de notes)
with open('C:/Users/maxim/OneDrive/Documents/GitHub/piano_project/Wav-Notes/notes_dict_1ligne.json', 'r') as f:
    data = json.load(f)
input_filepath = 'Wav-Notes/notes_dict_1ligne.json'
# Obtenir le repertoire du fichier original
output_directory = os.path.dirname(input_filepath)

# Parametres a ajuster
# liste filtre_bas : [100,100,150,150,200,200,250,250,300,300,350,350,400,400,500,500]
# liste filtre_haut : [2000,1500,2000,1500,2000,1500,2000,1500,2000,1500,2000,1500,2000,1500,2000,1500]
filtre_bas=[550]
filtre_haut=[1500]
redu=30 #facteur de reduction de la frequence d'echantillonnage

# Parametres importants
fs = int(44100//redu)  # sample rate
dt = 0.1  # Intervalle de temps (en secondes)
nb_recordings = 1  # nb d'enregistrements par note
nb_points = int(dt * fs)  # Equivalent en nombre de points pour les indices
point_du_tap = 11 # Sert a l'affichage

notes= ['1','2','3','4','5','6','7','8','9','10','11','12','13','14','15','16','17']
notes_matrix = np.zeros((len(notes) * nb_recordings, nb_points))

# Transferer les donnees du dictionnaire dans une matrice avec 17 lignes et nb_points par ligne
i = 0
for note, recordings in data.items():
    for prise in recordings:
        array = np.array(prise)
        array_fin=array[::redu]
        if len(array_fin)!=nb_points:
            array_fin=array_fin[:nb_points]
        notes_matrix[i] = array_fin
        i += 1

def dico_filtre_passband(matrice, frequence_minimale, frequence_maximale):
    # 1. Effectuer la transformee de Fourier rapide (FFT)
    signal_fft = np.fft.rfft(matrice)
    frequencies = np.fft.rfftfreq(len(matrice[0]), 1/fs)

    # 2. Creer un filtre passe-bande
    low_cutoff = frequence_minimale  # Frequence de coupure basse (Hz)
    high_cutoff = frequence_maximale  # Frequence de coupure haute (Hz)
    filter_mask = (frequencies > low_cutoff) & (frequencies < high_cutoff)

    # 3. Appliquer le filtre
    filtered_fft = signal_fft * filter_mask

    # 4. Revenir au domaine temporel avec la transformee inverse de Fourier
    filtered_signal = np.fft.irfft(filtered_fft)
    
    # 5. Remplacer la deuxieme moitie de chaque vecteur par des zeros
    filtered_signal_cut = np.zeros_like(filtered_signal)  # Creer une matrice du meme type, remplie de zeros
    half_point = filtered_signal.shape[1] // 2  # Obtenir le point de coupure (moitie)
    filtered_signal_cut[:, :half_point] = filtered_signal[:, :half_point]
    
    # Retourne le nouveau dico avec les signaux filtres
    return filtered_signal_cut, frequencies, filtered_fft[point_du_tap], signal_fft[point_du_tap]

# Fonction pour creer un nouveau fichier JSON pour chaque niveau de bits
def create_modified_json(fbas,fhaut,freq_sampling, output_directory):
    notes_dict = {}
    
    for bas,haut in zip(fbas,fhaut):
        matrice_traitee, frequencies, filtered_fft, signal_fft=dico_filtre_passband(notes_matrix,bas,haut)

    # On parcourt les notes et les enregistrements pour remplir le dictionnaire
    for i, note in enumerate(notes):
        recordings = []
        for j in range(nb_recordings):
            # On suppose que chaque enregistrement correspond a une ligne dans notes_matrix
            recordings.append(matrice_traitee[i * nb_recordings + j].tolist())  # Convertir en liste
        notes_dict[note] = recordings

    for bas,haut in zip(fbas,fhaut):
        # Generer le nom de fichier pour chaque duo de filtre
        output_filename = os.path.join(output_directory, f'fs={freq_sampling}-fbas={bas}-fhaut={haut}.json')

        # Sauvegarder le nouveau dictionnaire dans un fichier JSON
        with open(output_filename, 'w') as outfile:
            json.dump(notes_dict, outfile, indent=4)

# Creer un fichier JSON pour chaque niveau de bits dans le meme repertoire que le fichier original
create_modified_json(filtre_bas, filtre_haut, fs, output_directory)

#La suite du code n'est pas utile en soi mais permet de confirmer que ca donne la bonne chose
sign, frequencies, filtered_fft, signal_fft=dico_filtre_passband(notes_matrix,filtre_bas[0],filtre_haut[0])
signal_post_filtre=sign[point_du_tap]
signal=notes_matrix[point_du_tap]

# Generer le temps 
t1 = np.linspace(0, 1, len(signal), endpoint=False)  # Intervalle de temps
t2 = np.linspace(0, 1, len(signal_post_filtre), endpoint=False)  # Intervalle de temps
#f1, f2 = 50, 200  # Frequences des sinusoides
#signal = np.sin(2 * np.pi * f1 * t) + 0.5 * np.sin(2 * np.pi * f2 * t)

# Afficher les resultats
plt.figure(figsize=(10, 6))

# Affichage du signal original
plt.subplot(2, 2, 1)
plt.plot(t1, signal)
plt.title("Signal original")
plt.xlabel("Temps [s]")
plt.ylabel("Amplitude")

# Spectre de frequence original
plt.subplot(2, 2, 2)
plt.plot(frequencies[:fs//2], np.abs(signal_fft)[:fs//2])
plt.title("Spectre de frequence original")
plt.xlabel("Frequence [Hz]")
plt.ylabel("Amplitude")

# Spectre de frequence filtre
plt.subplot(2, 2, 4)
plt.plot(frequencies[:fs//2], np.abs(filtered_fft)[:fs//2])
plt.title("Spectre de frequence filtre")
plt.xlabel("Frequence [Hz]")
plt.ylabel("Amplitude")

# Signal filtre
plt.subplot(2, 2, 3)
plt.plot(t2, signal_post_filtre.real)
plt.title("Signal filtre")
plt.xlabel("Temps [s]")
plt.ylabel("Amplitude")

plt.tight_layout()
plt.show()
\end{lstlisting}

\subsection{Programme pour changer le nombre de bits des échantillons}

\begin{lstlisting}[language=python]
import json
import os
import numpy as np

# Charger le fichier JSON (dictionnaire)
input_filepath = 'Wav-Notes/notes_dict_1ligne.json'
with open(input_filepath, 'r') as f:
    data = json.load(f)

# Obtenir le repertoire du fichier original
output_directory = os.path.dirname(input_filepath)

# Fonction pour reduire la precision d'un signal en fonction du nombre de bits
def reduce_precision(signal, n_bits):
    if n_bits == 1:
        # Cas special pour 1 bit : toutes les valeurs negatives deviennent 0
        return [1 if value > 0 else 0 for value in signal]
    elif n_bits > 1:
        levels = 2**(n_bits - 1)  # Utiliser n_bits - 1 pour tenir compte du bit de signe
        # Quantifier le signal dans le nombre de niveaux approprie
        signal_quantified = np.round(np.array(signal) * (levels // 2)) / (levels // 2)
    else:
        signal_quantified = np.zeros_like(signal)  # Tout est ramene a zero pour 0 bit
    
    return signal_quantified.tolist()  # Convertir en liste pour garder le format JSON

# Fonction pour creer un nouveau fichier JSON pour chaque niveau de bits
def create_modified_json(data, n_bits, output_directory):
    # Creer un nouveau dictionnaire avec les signaux modifies
    modified_data = {}
    
    for note, vecteurs in data.items():
        modified_data[note] = [reduce_precision(vecteur, n_bits) for vecteur in vecteurs]
    
    # Generer le nom de fichier pour chaque niveau de bits
    output_filename = os.path.join(output_directory, f'modified_signal_{n_bits}bit.json')
    
    # Sauvegarder le nouveau dictionnaire dans un fichier JSON
    with open(output_filename, 'w') as outfile:
        json.dump(modified_data, outfile, indent=4)

# Creer un fichier JSON pour chaque niveau de bits dans le meme repertoire que le fichier original
create_modified_json(data, 16, output_directory)
create_modified_json(data, 8, output_directory)
create_modified_json(data, 6, output_directory)
create_modified_json(data, 4, output_directory)
create_modified_json(data, 3, output_directory)
create_modified_json(data, 2, output_directory)
create_modified_json(data, 1, output_directory)
create_modified_json(data, 0, output_directory)
\end{lstlisting}

\subsection{Programme pour calculer la résolution et le contraste pour un dictionnaire de notes}

\begin{lstlisting}[language=python]
import numpy as np
from scipy.odr import ODR, Model, RealData
import json
import pandas as pd
from scipy.optimize import curve_fit
import matplotlib.pyplot as plt
print('Librairies importees')

# liste des noms de dictionnaires JSON a charger
fichiers=['notes_dict_1ligne','modified_signal_1bit','modified_signal_2bit','modified_signal_3bit'
          ,'modified_signal_4bit','modified_signal_6bit','modified_signal_8bit','modified_signal_16bit']

# Charger les fichiers JSON
def lecteur():
    encyclopedie=[]
    for nom in fichiers:
        with open(f'Wav-Notes/{nom}.json', 'r') as f:
            encyclopedie.append(json.load(f))
    return encyclopedie 

# Fonction gaussienne avec floor ajustable pour curve_fit
def gaussian_with_floor(x, A, mu, sigma, floor):
    return A * np.exp(-((x - mu) ** 2) / (2 * sigma ** 2)) + floor

# Fonction pour ajuster avec des bornes (curve_fit)
def fit_gaussian_with_bounds(corr_data, xaxis, yerr):
    x_data = xaxis  # np.arange(len(corr_data))
    initial_guess = [np.max(corr_data), np.argmax(corr_data), np.std(corr_data), np.mean(corr_data)]  # [amplitude, mean, sigma, offset]

    # Contraintes sur les bornes pour que sigma > 0 et floor proche de la moyenne des donnees
    bounds = ([0, 0, 0, np.mean(corr_data) - 0.09], [np.inf, len(corr_data), np.inf, np.mean(corr_data) + 0.09])

    # Utilisation de curve_fit avec la nouvelle fonction gaussienne
    params, pcov = curve_fit(gaussian_with_floor, x_data, corr_data, p0=initial_guess,
                             sigma=yerr, absolute_sigma=True, bounds=bounds, maxfev=10000)

    perr = np.sqrt(np.diag(pcov))  # Erreurs sur les parametres ajustes
    return params, perr, x_data

# Fonction pour calculer la correlation du curve_fit
def correlateur_et_curve_fit_gaussien(data, point_du_tap):
        #Rembarque sur le code a Marielou
    # Parametres importants
    nb_points = len(data['1'][0])  
    dt = 0.1
    fs=int(nb_points/dt) # sample rate
    nb_recordings = 1  # nb d'enregistrements par note

    notes = ['1','2','3','4','5','6','7','8','9','10','11','12','13','14','15','16','17']
    matrice = np.zeros((len(notes) * nb_recordings, nb_points))

    i = 0
    for note, recordings in data.items():
        for prise in recordings:
            array = np.array(prise)
            array_normalise = array / np.max(array)
            matrice[i] = array_normalise
            i += 1

    signaux_reference = matrice
    signal = matrice[point_du_tap]
    position = 3*10**(-2) + 1.5*np.arange(0, 17)*10**(-2)
    correlation = np.dot(signaux_reference,signal)/np.max(np.dot(signaux_reference,signal))

    nb_bits = 32
    range = 2
    err_ampl = (range/(2**(nb_bits)))/2

    # Propagation de l'erreur sur le produit scalaire de la correlation
    yerr = []
    for element in signaux_reference:
        # Remplacer les zeros par une petite valeur pour eviter la division par zero
        signal_safe = np.where(signal == 0, 1e-10, signal)
        element_safe = np.where(element == 0, 1e-10, element)

        # Calculer l'erreur de multiplication
        err_multiplication = (signal_safe * element_safe) * np.sqrt((err_ampl / signal_safe) ** 2 + (err_ampl / element_safe) ** 2)

        # Verifier si err_multiplication contient des NaN ou des infinis
        if np.any(np.isnan(err_multiplication)) or np.any(np.isinf(err_multiplication)):
            print("Attention : err_multiplication contient des NaN ou des infinis.")
            continue  # Passer a l'iteration suivante

        # Calculer l'erreur totale
        err_prod = np.sqrt(np.sum(err_multiplication ** 2))
        yerr.append(err_prod)
    #yerr=1*correlation

    params, perr, x_data = fit_gaussian_with_bounds(correlation,position, yerr)
    
    amplitude_fit, mean_fit, sigma_fit, offset_fit = params
    amplitude_err, mean_err, sigma_err, offset_err = perr

    # La resolution est convertie en cm (chaque point est espace de 1,5 cm)
    resolution = 1.5 * np.log(2) * np.sqrt(2) * sigma_fit
    resolution_err = 1.5 * np.log(2) * np.sqrt(2) * sigma_err

    max_diff = amplitude_fit-offset_fit
    max_diff_err = amplitude_err-offset_err

    return {
        "resolution": resolution,
        "resolution_err": resolution_err,
        "max_diff": max_diff,
        "max_diff_err": max_diff_err,
    }

# Fonction gaussienne ODR
def gaussian_with_floor_constrained(p, x):
    A, mu, log_sigma, floor = p
    sigma = np.exp(log_sigma)  # sigma > 0 en utilisant la transformation exponentielle
    return A * np.exp(-((x - mu) ** 2) / (2 * sigma ** 2)) + floor

# Fonction pour calculer la correlation ODR
def correlateur_ODR(data,point_du_tap,nb_bit):
    # Parametres importants
    nb_points = len(data['1'][0])  
    dt = 0.1
    fs=int(nb_points/dt) # sample rate
    nb_recordings = 1  # nb d'enregistrements par note

    notes = ['1','2','3','4','5','6','7','8','9','10','11','12','13','14','15','16','17']
    matrice = np.zeros((len(notes) * nb_recordings, nb_points))

    i = 0
    for note, recordings in data.items():
        for prise in recordings:
            array = np.array(prise)
            array_normalise = array / np.max(array)
            matrice[i] = array_normalise
            i += 1

    signaux_reference = matrice
    signal = matrice[point_du_tap]
    position = 3*10**(-2) + 1.5*np.arange(0, 17)*10**(-2)
    correlation = np.dot(signaux_reference,signal)/np.max(np.dot(signaux_reference,signal))

    nb_bits = nb_bit
    range = 2
    err_ampl = (range/(2**(nb_bits)))/2
    err_position = 4e-3

    # Propagation de l'erreur sur le produit scalaire de la correlation
    err_prod_ampl = []
    for element in signaux_reference:
        # Remplacer les zeros par une petite valeur pour eviter la division par zero
        signal_safe = np.where(signal == 0, 1e-10, signal)
        element_safe = np.where(element == 0, 1e-10, element)

        # Calculer l'erreur de multiplication
        err_multiplication = (signal_safe * element_safe) * np.sqrt((err_ampl / signal_safe) ** 2 + (err_ampl / element_safe) ** 2)

        # Verifier si err_multiplication contient des NaN ou des infinis
        if np.any(np.isnan(err_multiplication)) or np.any(np.isinf(err_multiplication)):
            print("Attention : err_multiplication contient des NaN ou des infinis.")
            continue  # Passer a l'iteration suivante

        # Calculer l'erreur totale
        err_prod = np.sqrt(np.sum(err_multiplication ** 2))
        err_prod_ampl.append(err_prod)

    # Utilisation de ODR pour faire le fit gaussien avec sigma toujours positif
    data = RealData(position, correlation, sx=err_position, sy=err_prod_ampl)
    model = Model(gaussian_with_floor_constrained)
    
    guess_initial = [np.max(correlation), np.mean(position), np.log(np.std(position)), np.mean(correlation)]
    odr = ODR(data, model, beta0=guess_initial)
    output = odr.run()

    # Parametres optimaux et matrice de covariance
    A_opt, mu_opt, log_sigma_opt, floor_opt = output.beta
    sigma_opt = np.exp(log_sigma_opt)  # Revenir a sigma

    resolution = np.sqrt(2 * np.log(2)) * sigma_opt
    resolution_err = np.sqrt(2 * np.log(2)) * sigma_opt * np.sqrt(output.cov_beta[2, 2])

    contraste = A_opt-np.mean(correlation)
    contraste_err = 2*np.sqrt(output.cov_beta[0, 0])
    
    return {
        "resolution": resolution,
        "resolution_err": resolution_err,
        "max_diff": contraste,
        "max_diff_err": contraste_err,
    }

    # Ajuster les donnees avec la fonction gaussienne avec plancher
    params, perr, x_data = fit_gaussian_with_offset_and_errors(corr_data, yerr)
    
    # Extraction des parametres ajustes
    amplitude_fit, mean_fit, sigma_fit, offset_fit = params
    amplitude_err, mean_err, sigma_err, offset_err = perr

    # La resolution est convertie en cm (chaque point est espace de 1,5cm)
    resolution = 1.5 * np.log(2) * np.sqrt(2) * sigma_fit
    resolution_err = 1.5 * np.log(2) * np.sqrt(2) * sigma_err  

    # Calcul de la difference entre le maximum de la gaussienne et l'offset
    max_diff = amplitude_fit
    max_diff_err = amplitude_err  # Incertitude sur la difference est celle de l'amplitude

    # Retourner les resultats
    return {
        "resolution": resolution,
        "resolution_err": resolution_err,
        "max_diff": max_diff,
        "max_diff_err": max_diff_err,
    }

# Creer une liste pour stocker les resultats
results_list = []
dictionaries_to_process = lecteur()  # Ajoute les autres dictionnaires ici
bit_names = ['Original', '1bit', '2bit', '3bit', '4bit', '6bit', '8bit', '16bit']
bit_qty = [32, 1, 2, 3, 4, 6, 8, 16]  # bit_qty[idx]

for idx, current_data in enumerate(dictionaries_to_process):
    a1 = []
    a2 = []
    a3 = []
    a4 = []
    for i in [7, 8, 9, 10, 11]:
        corr_data = correlateur_ODR(current_data, i, bit_qty[idx])

        # Analyser les ajustements
        results = corr_data
        a1.append(results["resolution"])
        a2.append(results["resolution_err"])
        a3.append(results["max_diff"])
        a4.append(results["max_diff_err"])

    resolution = np.mean(a1)
    resolution_err = np.mean(a2)
    contraste = np.mean(a3)
    contraste_err = np.mean(a4)

    # Ajouter les resultats a la liste, en incluant les noms des bits
    results_list.append({
        "Dictionnaire": bit_names[idx],
        "Resolution": resolution,
        "Erreur Resolution": resolution_err,
        "Contraste": contraste,
        "Erreur Contraste": contraste_err,
    })

# Convertir les resultats en DataFrame
results_df = pd.DataFrame(results_list)

# Exporter les resultats en fichier Excel
results_df.to_excel('resultats_nouveaux_points_nb_bit.xlsx', index=False)

print("Analyse terminee et resultats exportes vers 'resultats'.")
\end{lstlisting}

\subsection{Programme pour enregistrer un dictionnaire de notes}

\begin{lstlisting}[language=python]
import sounddevice as sd
import numpy as np
import json

# temps pour taper la note
seconds = 2 

# freq d'acquisition par defaut
fs = 44100    

default = True # Si cette option est utilisee, le micro/speaker par defaut est utilise
devices = sd.query_devices()

if not default:
    InputStr = "Choisir le # correspondant au micro parmi la liste: \n"
    OutputStr = "Choisir le # correspondant au speaker parmi la liste: \n"
    for i in range(len(devices)):
        if devices[i]['max_input_channels']:
            InputStr += ('%d : %s \n' % (i, ''.join(devices[i]['name'])))
        if devices[i]['max_output_channels']:
            OutputStr += ('%d : %s \n' % (i, ''.join(devices[i]['name'])))
    DeviceIn = input(InputStr)
    DeviceOut = input(OutputStr)

    sd.default.device = [int(DeviceIn), int(DeviceOut)]

# liste de notes a enregistrer
notes = ['1','2','3','4','5','6','7','8','9','10','11','12','13','14','15','16','17']

notes_dict = {note: None for note in notes}
dt = 1e-1  # Intervalle de temps (en secondes)
nb_points = int(dt*fs) # Equivalent en nombre de points pour les indices

# enregistrement de chaque note
for note in notes:
    recordings = []
    print(f'Enregistrement de la note', note)

    # ecoute de la note
    myrecording = sd.rec(int(seconds * fs), samplerate=fs, channels=1)
    sd.wait()
    print(f'Enregistrement fini.')

    # Trouver l'amplitude maximale en valeur absolue
    max_amplitude = np.max(abs(myrecording))
    threshold = max_amplitude / 10  # Definir le seuil
    print(threshold)

    # Creer la fenetre utilisee pour le signal
    for index, value in enumerate(myrecording):
        if value >= threshold:
            start_signal = index
            break

    cut_signal = myrecording[start_signal:(start_signal + nb_points)].flatten()

    # Normalisation du signal
    norm_cut_signal = cut_signal / max_amplitude

    # Rajouter le nouveau array a la liste 'recordings'
    recordings.append(norm_cut_signal.tolist())

    # Transformer la liste en array
    nom_note = note
    notes_dict[nom_note] = recordings

# sauvegarde du dictionnaire en JSON
with open('notes_dict_1ligne.json', 'w') as json_file:
    json.dump(notes_dict, json_file)
\end{lstlisting}

\subsection{Programme pour jouer du piano en temps réel}

\begin{lstlisting}[language=python]
import sounddevice as sd
import numpy as np
import threading
import matplotlib.pyplot as plt
import time
from collections import deque
import pygame
from pydub import AudioSegment
import os
import json
import math

# chargement du dictionnaire
with open('Wav-Notes\\notes_dict_jo.json', 'r') as file:
    data = json.load(file)

# parametres audio
fs = 44100 
dt = 0.1  # Intervalle de temps (en secondes)
nb_recordings = 1 # Combien de signaux par note
nb_points = int(dt * fs)

# liste de notes
notes = ['1', '2', '3', '4', '5', '6', '7', '8', '9', '10', '11', '12', '13', '14', '15', '16', '17']
notes_matrix = np.zeros((len(notes) * nb_recordings, nb_points))

# Conversion des donnees en arrays et reformatage de la memoire
i = 0
for note, recordings in data.items():
    for element in recordings:
        array = np.array(element) 
        notes_matrix[i] = array
        i += 1

notes_dict = {
    '1': 'c3.wav',
    '2': 'c-3.wav',
    '3': 'd3.wav',
    '4': 'd-3.wav',
    '5': 'e3.wav',
    '6': 'f3.wav',
    '7': 'f-3.wav',
    '8': 'g3.wav',
    '9': 'g-3.wav',
    '10': 'a4.wav',
    '11': 'a-4.wav',
    '12': 'b4.wav',
    '13': 'c4.wav',
    '14': 'c-4.wav',
    '15': 'd4.wav',
    '16': 'd-4.wav',
    '17': 'e4.wav'
}

pygame.mixer.init()

# THREAD : fonction pour jouer l'audio d'une note
def jouer_note(note):
    if note in notes_dict:
        fichier_note = f'Wav-Notes\\{notes_dict[note]}'
        
        if os.path.isfile(fichier_note):
            # Load the WAV file in memory
            note_sound = AudioSegment.from_wav(fichier_note)
            
            # Play the note using pygame
            son = pygame.mixer.Sound(fichier_note)
            son.play()
            pygame.time.wait(int(note_sound.duration_seconds * 1000))  # attendre fin de l'audio
        else:
            print(f"Le fichier {fichier_note} n'existe pas.")
    else:
        print("Note non reconnue.")

# test de note
jouer_note('c3')

# parametres de detection de signal audio
threshold = 0.01  # amplitude minimale pour signaler une impulsion
spike_detected = False  # flag pour savoir si un signal est detecte
capture_duration = 0.15  # temps de capture d'audio post impulsion
buffer_size = fs  # taille du buffer qui contient le signal audio roulant
signal_buffer = np.zeros(buffer_size)

# Au besoin, fonction pour visualiser le signal audio d'une impulsion
def plot_data(data):
    t = np.linspace(0, dt, len(data))
    plt.plot(t, data)
    plt.xlabel('Time [s]')
    plt.ylabel('Amplitude')
    plt.title('Signal After Spike')
    plt.show()

# Lecture d'audio en temps reel
def audio_callback(indata, frames, time, status):
    global spike_detected, signal_buffer
    
    if status:
        print(status)
    
    # aplatissage du data entrant
    audio_data = indata[:, 0]
    
    # roulement du signal dans le buffer
    signal_buffer = np.roll(signal_buffer, -frames)
    signal_buffer[-frames:] = audio_data
    
    # Analyse du buffer pour l'impulsion
    if not spike_detected and np.max(audio_data) > threshold:
        spike_detected = True
        print("Spike detected!")
    
        # Si detection, lance le thread d'analyse de signal
        capture_thread = threading.Thread(target=signal_analysis)
        capture_thread.start()

# THREAD : Analyse de signal
def signal_analysis():
    global spike_detected
    
    time.sleep(capture_duration)  # laisse le buffer prendre la suite du signal post impulsion

    # Capture du signal de l'impulsion dans le buffer
    post_spike_data = np.copy(signal_buffer)[int(fs - fs * capture_duration - 800):]
    
##################################################################
# Traitement de donnees
##################################################################

    data_max_amp = np.max(abs(post_spike_data))
    data_threshold = data_max_amp / 10

    # Creer la fenetre utilisee pour le signal
    for index, value in enumerate(post_spike_data):
        if value >= data_threshold:
            start_signal = index
            break
    cut_data = post_spike_data[start_signal:(start_signal + int(dt * fs))].flatten()

    # Plot the data
    # plot_data(cut_data)

    # Normalisation du signal
    norm_cut_data = cut_data / data_max_amp

    # Transformer la liste en array
    signal_array = np.array(norm_cut_data)

    # Produit scalaire (correlation) entre les donnees de training et le signal test
    scalar_prod = np.dot(notes_matrix, signal_array)

    # Trouver l'indice de la valeur max du produit scalaire et trouver sa note correspondante
    index_max = np.argmax(scalar_prod)
    note_index = index_max // nb_recordings
    print(scalar_prod)

    note = notes[note_index]
    if note:
        print(f"Playing note: {note}")
        # Lance le thread pour jouer la note identifiee
        play_note_thread = threading.Thread(target=jouer_note, args=(note,))
        play_note_thread.start()
    
    # Reset du flag pour continuer a lire des impulsions
    spike_detected = False

# Lancement de la lecture d'audio en continu
with sd.InputStream(callback=audio_callback, samplerate=fs, channels=1):
    print("Recording... (Press Ctrl+C to stop)")
    while True:
        time.sleep(0.001)  # Keep the main loop running
\end{lstlisting}


\clearpage

% \bibliographystyle{unsrtnat}
% \bibliography{My_Library}

\end{document}
