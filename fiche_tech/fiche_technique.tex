\documentclass[11pt,letterpaper]{article}
\usepackage[top=3cm, bottom=2cm, left=2cm, right=2cm, columnsep=20pt]{geometry}
\usepackage{pdfpages}
\usepackage{graphicx}
\usepackage{etoolbox}
\apptocmd{\sloppy}{\hbadness 10000\relax}{}{}
% \usepackage[numbers]{natbib}
\usepackage[T1]{fontenc}
\usepackage{ragged2e}
\usepackage[french]{babel}
\usepackage{listings}
\usepackage{color}
\usepackage{soul}
\usepackage[utf8]{inputenc}
\usepackage[export]{adjustbox}
\usepackage{caption}
\usepackage{amsmath}
\usepackage{amssymb}
\usepackage{float}
\usepackage{csquotes}
\usepackage{fancyhdr}
\usepackage{wallpaper}
\usepackage{siunitx}
\usepackage[indent]{parskip}
\usepackage{textcomp}
\usepackage{gensymb}
\usepackage{multirow}
\usepackage[hidelinks]{hyperref}
\usepackage{abstract}
\usepackage{svg}
\renewcommand{\abstractnamefont}{\normalfont\bfseries}
\renewcommand{\abstracttextfont}{\normalfont\itshape}
\usepackage{titlesec}
\titleformat{\section}{\large\bfseries}{\thesection}{1em}{}
\titleformat{\subsection}{\normalsize\bfseries}{\thesubsection}{1em}{}
\titleformat{\subsubsection}{\normalsize\bfseries}{\thesubsubsection}{1em}{}

\usepackage{xcolor}
\definecolor{codegreen}{rgb}{0,0.6,0}
\definecolor{codegray}{rgb}{0.5,0.5,0.5}
\definecolor{codepurple}{rgb}{0.58,0,0.82}
\definecolor{backcolour}{rgb}{0.95,0.95,0.92}
\lstdefinestyle{mystyle}{
    backgroundcolor=\color{backcolour},   
    commentstyle=\color{codegreen},
    keywordstyle=\color{magenta},
    numberstyle=\tiny\color{codegray},
    stringstyle=\color{codepurple},
    basicstyle=\ttfamily\footnotesize,
    breakatwhitespace=false,         
    breaklines=true,                 
    captionpos=b,                    
    keepspaces=true,                 
    numbers=left,                    
    numbersep=5pt,                  
    showspaces=false,                
    showstringspaces=false,
    showtabs=false,                  
    tabsize=2
}
\lstset{style=mystyle}

\usepackage[most]{tcolorbox}
\newtcolorbox{note}[1][]{
  enhanced jigsaw,
  borderline west={2pt}{0pt}{black},
  sharp corners,
  boxrule=0pt, 
  fonttitle={\large\bfseries},
  coltitle={black},
  title={Note:\ },
  attach title to upper,
  #1
}

%----------------------------------------------------

\setlength{\parindent}{0pt}
\DeclareCaptionLabelFormat{mycaptionlabel}{#1 #2}
\captionsetup[figure]{labelsep=colon}
\captionsetup{labelformat=mycaptionlabel}
\captionsetup[figure]{name={Figure }}
\newcommand{\inlinecode}{\normalfont\texttt}
\usepackage{enumitem}
\setlist[itemize]{label=\textbullet}

\begin{document}
\begin{titlepage}
\center

\begin{figure}
    \ThisULCornerWallPaper{.4}{Polytechnique_signature-RGB-gauche_FR.png}
\end{figure}
\vspace*{2 cm}

\textsc{\Large \textbf{PHS3910 --} Techniques expérimentales et instrumentation}\\[0.5cm]
\large{\textbf{Équipe : Lundi 03}}\\[1.5cm]

\rule{\linewidth}{0.5mm} \\[0.5cm]
\Large{\textbf{Écran tactile acoustique}} \\[0.2cm]
\text{Fiche technique du prototype}\\
\rule{\linewidth}{0.2mm} \\[2.3cm]

\large{\textbf{Présenté à}\\
  Jean Provost\\
  Lucien Weiss\\[2.5cm]
  \textbf{Par :}\\
  Émile \textbf{Guertin-Picard} (2208363)\\
  Philippine \textbf{Beaubois} (2211153)\\
  Marie-Lou \textbf{Dessureault} (2211129)\\
  Maxime \textbf{Rouillon} (2213291)\\[3cm]}

\large{\today\\
Département de Génie Physique\\
Polytechnique Montréal\\}

\end{titlepage}

%----------------------------------------------------

\tableofcontents
\pagenumbering{roman}
\newpage

\pagestyle{fancy}
\setlength{\headheight}{14pt}
\renewcommand{\headrulewidth}{0pt}
\fancyfoot[R]{\thepage}

\pagestyle{fancy}
\fancyhf{}
\renewcommand{\headrulewidth}{1pt}
\fancyhead[L]{\textbf{PHS3910}}
\fancyhead[C]{Fiche technique de l'écran tactile acoustique}
\fancyhead[R]{\today}
\fancyfoot[R]{\thepage}

\pagenumbering{arabic}
\setcounter{page}{1}

%----------------------------------------------------

\section{Description générale}

%%%%%%%%%%%%%%%%%%%%%%%%%%%%%%%%%%%%%%%%%%%%%%%%%%%%
% Idées :

% Nombre de notes, un seul octave, une note à la fois.

%%%%%%%%%%%%%%%%%%%%%%%%%%%%%%%%%%%%%%%%%%%%%%%%%%%%

Cette fiche technique présente les caractéristiques d'un piano construit avec un
écran tactile acoustique. Un capteur piezoélectrique, placé sur une plaque de
plexiglas de 5 mm d'épaisseur, permet de localiser un impact sur la plaque par 
son onde sonore, pour permettre ensuite de jouer la note appropriée. Cette plaque 
et ses dimensions sont présentées à la figure \ref{piano_fig}. Le piano peut jouer\
une seule gamme (12 notes), et est limité à ne pouvoir jouer qu'une seule note à 
la fois. L'acquisition de signal sonore se fait à une fréquence de 44100 Hz par un
ADC, donnant des échantillons de 32 bits. Le contenu fréquentiel n'est pas filtré,
donc des fréquences sonores de 0 Hz à 22000 Hz sont présentes. Le piano joue en temps
réel, avec un délai entre la frappe et le son de la note d'approximativement 200 ms,
dépendant de la puissance de l'ordinateur qui lit les données du capteur. \textcolor{red}{mettre res et contraste une fois connues}

\begin{figure}[H]
  \centering
  \includesvg[scale=1]{schema.svg}
  \caption{Schéma avec dimensions du prototype de piano tactile. Le capteur
  piezoélectrique a son centre de positionné à 176 mm du côté gauche et à 207 mm
  du bas de la plaque approximativement.}
  \label{piano_fig}
\end{figure}



\section{Spécifications}

dire pourquoi on a testé res et contraste pour d'autres facteurs que ceux du prototype réel (aka pour l'analyse de réduction de couts)
%%%%%%%%%%%%%%%%%%%%%%%%%%%%%%%%%%%%%%%%%%%%%%%%%%%%
% Temps de réponse moyen serait nice

% Res et contraste :
% nb de bits des échantillons du signal (int16, float32 hors de la carte)
% contenu fréquentiel
% fréquence d'échantillonage

% Forme de tableau
%%%%%%%%%%%%%%%%%%%%%%%%%%%%%%%%%%%%%%%%%%%%%%%%%%%%


\section{Rapports de tests}

%%%%%%%%%%%%%%%%%%%%%%%%%%%%%%%%%%%%%%%%%%%%%%%%%%%%
% description brève des tests et présentations de leurs résultats.

% si graphiques : plusieurs courbes et infos sur le même

% Tests :

% acquisition des résolutions et contrastes pour les 3 facteurs

% *** écrire en qqpart l'indépendance du param de nb de bits vs les autres ***
%%%%%%%%%%%%%%%%%%%%%%%%%%%%%%%%%%%%%%%%%%%%%%%%%%%%

Pour pouvoir réduire les coûts blabla... \textcolor{red}{TODO}
\subsection{bite}
 
\subsection{Incertitudes}
L'incertitude associée à l'amplitude du signal, ou l'axe y, est celle qui provient de la résolution 
de l'ADC. La résolution de l'ADC, $\delta$, dépend du nombre de bits des échantillons, et peut être décrite par:
\[\delta=\frac{\Delta_{max}}{2^{n}},\]
où $\Delta_{max}$ est l'éventail possible des mesures, et $n$ est le nombre de bits. L'incertitude correspond
à la moitié de cette valeur, soit $\sigma_{res}=\delta/2$. L'incertitude sur la position de la source du signal
est approximée par la moitié de la largeur d'un doigt, $\sigma_{pos}\approx4\ mm$. La corrélation des signaux a été déterminée
en calculant le produit scalaire. On sait que pour une multiplication, la propagation de l'incertitude se calcule de la
manière suivante:
\begin{align*}
  \sigma_{z}=z \sqrt{\frac{\sigma_x}{x}^2+\frac{\sigma_y}{y}^2}.
\end{align*}
L'incertitude totale sur le produit scalaire est donc:
\begin{align*}
  \sigma_{tot}=\sqrt{\sum_i^{N} \sigma_{z_i}^2},
\end{align*}
où $N$ est le nombre de points évalués pour les échantillons. Pour déterminer l'incertitude sur les
paramètres du fit gaussien, tout en considérant l'impact des incertitudes en $x$ et $y$, la fonction \textit{scipy.ODR}
a été utilisée. Finalement, l'incertitude sur le contraste correspond directement à l'incertitude sur le paramètre de
l'amplitude ($\sigma_{A}$), et l'incertitude sur la résolution (FWHM) correspond à:
\begin{align*}
  \sigma_{FWHM}=\sqrt{2ln2}\ \sigma_{\sigma},
\end{align*}
où $\sigma_{\sigma}$ est l'incertitude sur l'écart-type, $\sigma$.
\section{Codes}

 %Important de clean le code avant de remettre, j'ai entendu dire qu'il était picky là-dessus.



\clearpage

% \bibliographystyle{unsrtnat}
% \bibliography{My_Library}

\end{document}
