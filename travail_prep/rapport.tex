\documentclass[conference]{IEEEtran}
\usepackage[top=3cm, bottom=2cm, left=2cm, right=2cm, columnsep=20pt]{geometry}
\usepackage{pdfpages}
\usepackage{graphicx}
\usepackage{etoolbox}
\apptocmd{\sloppy}{\hbadness 10000\relax}{}{}
% \usepackage[numbers]{natbib}
\usepackage[T1]{fontenc}
\usepackage{ragged2e}
\usepackage[french]{babel}
\usepackage{listings}
\usepackage{color}
\usepackage{soul}
\usepackage[utf8]{inputenc}
\usepackage[export]{adjustbox}
\usepackage{caption}
\usepackage{amsmath}
\usepackage{amssymb}
\usepackage{float}
\usepackage{csquotes}
\usepackage{fancyhdr}
\usepackage{wallpaper}
\usepackage{siunitx}
\usepackage[indent]{parskip}
\usepackage{textcomp}
\usepackage{gensymb}
\usepackage{multirow}
\usepackage[hidelinks]{hyperref}
\usepackage{abstract}
% \renewcommand{\abstractnamefont}{\normalfont\bfseries}
% \renewcommand{\abstracttextfont}{\normalfont\itshape}
\usepackage{titlesec}
% \titleformat{\section}{\large\bfseries}{\thesection}{1em}{}
% \titleformat{\subsection}{\normalsize\bfseries}{\thesubsection}{1em}{}
% \titleformat{\subsubsection}{\normalsize\bfseries}{\thesubsubsection}{1em}{}

\usepackage{xcolor}
\definecolor{codegreen}{rgb}{0,0.6,0}
\definecolor{codegray}{rgb}{0.5,0.5,0.5}
\definecolor{codepurple}{rgb}{0.58,0,0.82}
\definecolor{backcolour}{rgb}{0.95,0.95,0.92}
\lstdefinestyle{mystyle}{
    backgroundcolor=\color{backcolour},   
    commentstyle=\color{codegreen},
    keywordstyle=\color{magenta},
    numberstyle=\tiny\color{codegray},
    stringstyle=\color{codepurple},
    basicstyle=\ttfamily\footnotesize,
    breakatwhitespace=false,         
    breaklines=true,                 
    captionpos=b,                    
    keepspaces=true,                 
    numbers=left,                    
    numbersep=5pt,                  
    showspaces=false,                
    showstringspaces=false,
    showtabs=false,                  
    tabsize=2
}
\lstset{style=mystyle}

\usepackage[most]{tcolorbox}
\newtcolorbox{note}[1][]{
  enhanced jigsaw,
  borderline west={2pt}{0pt}{black},
  sharp corners,
  boxrule=0pt, 
  fonttitle={\large\bfseries},
  coltitle={black},
  title={Note:\ },
  attach title to upper,
  #1
}

%----------------------------------------------------

\setlength{\parindent}{0pt}
\DeclareCaptionLabelFormat{mycaptionlabel}{#1 #2}
\captionsetup[figure]{labelsep=colon}
\captionsetup{labelformat=mycaptionlabel}
\captionsetup[figure]{name={Figure }}
\captionsetup[table]{name=Tableau}
\newcommand{\inlinecode}{\normalfont\texttt}
\usepackage{enumitem}
\setlist[itemize]{label=\textbullet}

\begin{document}

%----------------------------------------------------
\title{Écran tactile acoustique\\
\large Travail préparatoire \\
PHS3910 -- Techniques expérimentales et instrumentation\\ 
Équipe L3}

\author{\IEEEauthorblockN{Émile Guertin-Picard}
\IEEEauthorblockA{2208363}
\and
\IEEEauthorblockN{Maxime Rouillon}
\IEEEauthorblockA{2213291}
\and
\IEEEauthorblockN{Marie-Lou Dessureault}
\IEEEauthorblockA{2211129}
\and
\IEEEauthorblockN{Philippine Beaubois}
\IEEEauthorblockA{2211153}
}

\maketitle

\section{Introduction}
Dans le cadre du cours PHS3910, l'équipe est mandatée de conceptualiser
un piano tactile à l'aide du principe de retournement temporel. Avant
de développer le produit final, il faut premièrement simuler le fonctionnement 
du piano à l'aide de l'outil k-Wave, afin de déterminer les caractéristiques qui 
optimiseront sa performance. En variant la forme, le matériau de 
la plaque ainsi que la position du capteur directement dans la simulation,
une solution complète préliminaire peut être établie. Le choix de la solution finale est
principalement basé sur les concepts de résolution et de contraste.

Les choix retenus suite aux tests de simulation sont les suivants: une plaque
asymétrique avec trous, en aluminium, avec un capteur situé \textcolor{red}{??}.
Le rapport ci-présent détaillera la méthodologie utilisée afin d'obtenir 
une étude statistique de la résolution et du contraste en fonction des
caractéristiques du piano, présentera les valeurs résultantes et leurs incertitudes 
respectives, et discutera des éléments pouvant être déduis de ceux-ci et implémentés 
dans la conception. Une présentation des prochaines étapes nécessaires à la complétion
du mandat sera aussi détaillée.

\section{Méthodes}
%Garder en tête qu'on veut ''On vous demande de déterminer et de 
%faire une étude systémique et rigoureuse des paramètres
%principaux à optimiser. ''

%''piano tactile le plus près possible d’un piano acoustique''

%Méthodes : Transformer cette question dans un langage mathématique, 
%décrire les méthodes de simulations
Les géométries des plaques considérées ont été choisies pour avoir un éventail de 
symétries (carré, cercle, forme asymétrique). Des simulations ont aussi été effectuées 
sur des plaques avec et sans trous. Du bois, du plexiglas et des métaux 
ont été testés pour voir l'impact de la vitesse du son dans le milieu sur
le contraste et la résolution. Plus d'attention a été portée sur les métaux,
sachant déjà que les matériaux plus denses et moins élastiques sont favorables à 
la transmission d'ondes acoustiques \textcolor{red}{ajouter reference}.

Étant donné que le problème à considérer est de trouver les paramètres 
optimaux à l'aide d'une simulation, les méthodes de simulations seront décrites
en fonction des étapes importantes du programme MatLab. Au lieu de simuler une source
pour un grand nombre de positions sur chaque plaque et pour chaque matériau, il a plutôt
été convenu d'effectuer la simulation d'un seul emplacement, et de faire usage de l'utilité
du principe de retournement temporel. Effectivement, seulement un impact a eu besoin
d'être simulé à un endroit arbitraire sur chaque plaque, mais ce, en évaluant le signal reçu
à une multitude de récepteurs distribués sur la plaque. Les récepteurs agissent alors en tant
que sources et le signal reçu au récepteur par chacune d'entre elles devient:
\[e(t)=h_{P_jS}(-t).\]
Pour réduire le temps de simulation, les retards accumulés sur chaque signal ont été 
retirés en gardant seulement une fenêtre de la réponse impulsionnelle. Les signaux inversés et 
recadrés ont été stockés et enregistrés pour pouvoir s'y référer lors du calcul de la corrélation.

%Étapes AU PASSÉÉÉ
%1 Créer les configurations (Matériau + Géométrie)
%2 Simuler un impact à un endroit avec K-wave
%3 Évaluer le signal pour des récepteurs positionnés partout sur le matériau
%4 Faire un retournement temporel sur tous les signaux. Ce processus fiat des récepteurs les émeteurs, et de l'émetteur le récepteur de chaque émission
%5 Éliminer les retards à l'aide d'une fonction fenêtre/rectangle pour accélérer la comparaison des signaux
%6 Pour gagner du temps, on choisi une bande d'intérêt de la plaque où on sélectionne les émissions une à la fois. 
%7 Pour chaque émission, on la compare à toutes les émissions enregistrées de la bande, formant un graphique 2D de la corrélation en fonction de la position
%8 Une fois chaque émission de la bande d'intérêt testée, on a effectué une étude statistique de la résolution et du constraste en ''fittant'' une gaussienne sur le graphique 2D. 


%Observer les résultats sur une seule ligne pour ne pas saturer les 
%capacités de calcul. 

%Analyse du signal au récepteur, ce qui correspond à la réponse 
%impulsionnel de la configuration testée à une impulsion à 
%l'emplacement de la source. 
%+ de description mathématique




\section{Résultats}


\section{Discussion}
\clearpage

% \bibliographystyle{unsrtnat}
% \bibliography{My_Library}

\end{document}
