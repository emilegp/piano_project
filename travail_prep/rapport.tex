\documentclass[conference]{IEEEtran}
\usepackage[top=3cm, bottom=2cm, left=2cm, right=2cm, columnsep=20pt]{geometry}
\usepackage{pdfpages}
\usepackage{graphicx}
\usepackage{etoolbox}
\apptocmd{\sloppy}{\hbadness 10000\relax}{}{}
% \usepackage[numbers]{natbib}
\usepackage[T1]{fontenc}
\usepackage{ragged2e}
\usepackage[french]{babel}
\usepackage{listings}
\usepackage{color}
\usepackage{soul}
\usepackage[utf8]{inputenc}
\usepackage[export]{adjustbox}
\usepackage{caption}
\usepackage{amsmath}
\usepackage{amssymb}
\usepackage{float}
\usepackage{csquotes}
\usepackage{fancyhdr}
\usepackage{wallpaper}
\usepackage{siunitx}
\usepackage[indent]{parskip}
\usepackage{textcomp}
\usepackage{gensymb}
\usepackage{multirow}
\usepackage[hidelinks]{hyperref}
\usepackage{abstract}
% \renewcommand{\abstractnamefont}{\normalfont\bfseries}
% \renewcommand{\abstracttextfont}{\normalfont\itshape}
\usepackage{titlesec}
% \titleformat{\section}{\large\bfseries}{\thesection}{1em}{}
% \titleformat{\subsection}{\normalsize\bfseries}{\thesubsection}{1em}{}
% \titleformat{\subsubsection}{\normalsize\bfseries}{\thesubsubsection}{1em}{}

\usepackage{xcolor}
\definecolor{codegreen}{rgb}{0,0.6,0}
\definecolor{codegray}{rgb}{0.5,0.5,0.5}
\definecolor{codepurple}{rgb}{0.58,0,0.82}
\definecolor{backcolour}{rgb}{0.95,0.95,0.92}
\lstdefinestyle{mystyle}{
    backgroundcolor=\color{backcolour},   
    commentstyle=\color{codegreen},
    keywordstyle=\color{magenta},
    numberstyle=\tiny\color{codegray},
    stringstyle=\color{codepurple},
    basicstyle=\ttfamily\footnotesize,
    breakatwhitespace=false,         
    breaklines=true,                 
    captionpos=b,                    
    keepspaces=true,                 
    numbers=left,                    
    numbersep=5pt,                  
    showspaces=false,                
    showstringspaces=false,
    showtabs=false,                  
    tabsize=2
}
\lstset{style=mystyle}

\usepackage[most]{tcolorbox}
\newtcolorbox{note}[1][]{
  enhanced jigsaw,
  borderline west={2pt}{0pt}{black},
  sharp corners,
  boxrule=0pt, 
  fonttitle={\large\bfseries},
  coltitle={black},
  title={Note:\ },
  attach title to upper,
  #1
}

%----------------------------------------------------

\setlength{\parindent}{0pt}
\DeclareCaptionLabelFormat{mycaptionlabel}{#1 #2}
\captionsetup[figure]{labelsep=colon}
\captionsetup{labelformat=mycaptionlabel}
\captionsetup[figure]{name={Figure }}
\captionsetup[table]{name=Tableau}
\newcommand{\inlinecode}{\normalfont\texttt}
\usepackage{enumitem}
\setlist[itemize]{label=\textbullet}

\begin{document}

%----------------------------------------------------
\title{Écran tactile acoustique\\
\large Travail préparatoire \\
PHS3910 -- Techniques expérimentales et instrumentation\\ 
Équipe L3}

\author{\IEEEauthorblockN{Émile Guertin-Picard}
\IEEEauthorblockA{2208363}
\and
\IEEEauthorblockN{Maxime Rouillon}
\IEEEauthorblockA{2213291}
\and
\IEEEauthorblockN{Marie-Lou Dessureault}
\IEEEauthorblockA{2211129}
\and
\IEEEauthorblockN{Philippine Beaubois}
\IEEEauthorblockA{2211153}
}

\maketitle

\section{Introduction}
Dans le cadre du cours PHS3910, l'équipe est mandatée de conceptualiser
un piano tactile à l'aide du principe de retournement temporel. Avant
de développer le produit final, il faut premièrement simuler le fonctionnement 
du piano à l'aide de l'outil k-Wave, afin de déterminer les caractéristiques qui 
optimiseront sa performance. En variant la forme, le matériau de 
la plaque ainsi que la position du capteur directement dans la simulation,
une solution complète préliminaire peut être établie. Le choix de la solution finale est
principalement basé sur les concepts de résolution et de contraste.

Les choix retenus suite aux tests de simulation sont présentés dans le tableau
\ref{tableau:1}.
\begin{table}[h!]
\centering
  \begin{tabular}{|c|c|} 
   \hline
   Forme & Asymétrique avec trous\\ 
   \hline
   Matériau & \\ 
   \hline
   Position du capteur & \\ 
   \hline
  \end{tabular}
  \caption{Choix préliminaires pour la conception du piano tactile}
  \label{tableau:1}
\end{table}
Le rapport ci-présent détaillera la méthodologie utilisée afin d'obtenir 
une étude statistique de la résolution et du contraste en fonction des
caractéristiques du piano, présentera les valeurs résultantes et leurs incertitudes 
respectives, et discutera des éléments pouvant être déduis de ceux-ci et implémentés 
dans la conception. Une présentation des prochaines étapes nécessaires à la complétion
du mandat sera aussi détaillée.

\section{Méthodes}
%Garder en tête qu'on veut ''On vous demande de déterminer et de 
%faire une étude systémique et rigoureuse des paramètres
%principaux à optimiser. ''

%''piano tactile le plus près possible d’un piano acoustique''

%Méthodes : Transformer cette question dans un langage mathématique, 
%décrire les méthodes de simulations
Les géométries des plaques considérées ont été choisies pour avoir un éventail de 
symétries (carré, forme asymétrique). Des simulations ont aussi été effectuées 
sur des plaques avec et sans trous. Du bois, du plexiglas et des métaux 
ont été testés pour voir l'impact de la vitesse du son dans le milieu sur les
le contraste et la résolution. Plus d'attention a été portée sur les métaux,
sachant déjà que les matériaux plus denses et moins élastiques sont favorables à 
la transmission d'ondes acoustiques.

Étant donné que le problème à considérer est de trouver les paramètres 
optimaux à l'aide d'une simulation, les méthodes de simulations seront décrites
en fonction des étapes importantes du programme MatLab. Au lieu de simuler une source
pour un grand nombre de positions sur chaque plaque et pour chaque matériau, il a plutôt
été convenu d'effectuer la simulation



%Pour tester de nombreuses configurations rapidement et facilement, 
%un gabarit de code MatLab a été utilisé pour simuler la propagation 
%de l'onde sonore dans le matériau du piano tactile. 
%Pour se faire, une fonction a été développée pour simuler un piano 
%d'un matériau homogène et d'un géométrie définie, avec un émetteur 
%et un récepteur à une position déterminée. La fonction permet de 
%varier le matériau, la géométrie et les positions de l'émetteur et 
%du récepteur. 
%Pour chaque matériau et tests de différentes positions dans ces 
%configurations...

%Observer les résultats sur une seule ligne pour ne pas saturer les 
%capacités de calcul. 

%Analyse du signal au récepteur, ce qui correspond à la réponse 
%impulsionnel de la configuration testée à une impulsion à 
%l'emplacement de la source. 
%+ de description mathématique




\section{Résultats}


\section{Discussion}
\clearpage

% \bibliographystyle{unsrtnat}
% \bibliography{My_Library}

\end{document}
